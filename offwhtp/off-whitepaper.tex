\documentclass[10pt,a4paper,leqno,bibliography=totoc]{scrartcl}
\usepackage[utf8]{inputenc}
\usepackage{lastpage} % Required to print the total number of pages
\usepackage[left=1.3cm,right=1.3cm,top=1.8cm,bottom=4.0cm]{geometry} % Adjust page margins
\usepackage{amsmath} % Required for equation customization
\usepackage{amssymb} % Required to include mathematical symbols
\usepackage{makecell}
\usepackage{forest}
\usepackage{boldline}
\usepackage{beton}
\usepackage{booktabs}
\usepackage[T1]{fontenc}
\usepackage[normalem]{ulem}
\usepackage{array}
\usepackage[colorlinks]{hyperref}
\usepackage{graphicx}
\usepackage{multicol}
\usepackage{setspace}
\usepackage[title,titletoc]{appendix}
\usepackage{mathtools}
\usepackage{subfig}
\usepackage{ragged2e}
\usepackage[singlelinecheck=false]{caption}
\usepackage{wrapfig}
\usepackage{IEEEtrantools}
\usepackage{lipsum}
\usepackage[english]{babel}
\usepackage{blindtext}
\usepackage{pgf}
\usepackage{tikz}
\usetikzlibrary{mindmap,arrows,automata,shadows}
\usepackage{epigraph}
\usepackage{tabularx}
\usepackage[perpage]{footmisc}
\usepackage[style=ieee,
	backref=true
	]{biblatex}
\usepackage{etoolbox}
\usepackage{framed,color}
\usepackage[framemethod=tikz]{mdframed}
\usepackage{minitoc}
\usepackage{fancybox}
\usepackage{tablefootnote}
\usepackage{datatool}
\usepackage{footnote}
\usepackage{pgfkeys}
\usepackage{pgf}
\usepackage{microtype}
\usepackage{enumitem}
\usepackage[acronyms,toc,section=section]{glossaries}
\usepackage{endnotes}
\usepackage{dirtytalk}
\usepackage{pdfpages}
\usepackage{supertabular}
\usepackage{kantlipsum}
\usepackage{makeidx}
\makeindex
\usepackage[totoc]{idxlayout}
\usepackage{longtable}
\usepackage{fancyhdr}
\usepackage{array}
\newcolumntype{P}[1]{>{\centering\arraybackslash}p{#1}}

\pagestyle{fancy}
\fancyhf{}
\fancyhead[C]{Beigepaper}
\fancyhead[L]{\rightmark}
\fancyhead[R]{\today--Version 0.9.5}
\fancyfoot[C]{\thepage} 
\renewcommand{\headrulewidth}{0.5pt}
\renewcommand{\footrulewidth}{0.5pt}
\renewcommand{\sectionmark}[1]{\markright{\thesection.\ #1}}

\glstoctrue
\newmdenv[shadow=true,shadowcolor=black,font=\sffamily,rightmargin=8pt]{shadedbox}

\definecolor{beige}{RGB}{245,245,220}

\newcount\n
\n=0
\def\tablebody{}
\makeatletter
\loop\ifnum\n<100
 \advance\n by1
 \protected@edef\tablebody{\tablebody
 \textbf{\number\n.}& shortText
 \tabularnewline
  }
\repeat
\makeatletter
			\let\mcnewpage=\newpage
		\newcommand{\TrickSupertabularIntoMulticols}{%
\renewcommand\newpage{%
	      \if@firstcolumn
	            \hrule width\linewidth height0pt
          \columnbreak
      \else
        \mcnewpage
       \fi
  }%
 }
 \makeatother

%╔═══════════════════DELETED TERMS═════════════════════╗
%║Autonomous Object				       ║
%║Design Pattern				       ║
%║Object-Oriented Programming			       ║
%║External Actor				       ║
%║Abstract Machine				       ║
%║Hacker Ethic					       ║
%╚═════════════════════════════════════════════════════╝

\makenoidxglossaries
\setacronymstyle{long-short}
\newacronym{EVM}{EVM}{Ethereum Virtual Machine}
\newacronym{ERE}{ERE}{Ethereum Runtime Environment}
\newacronym{RLP}{RLP}{Recursive Length Prefix}
\newglossaryentry{serialization}{name={serialization}, description={Serialization is the process of converting an object into a stream of bytes in order to store the object or transmit it to memory, a database, or a file. Its main purpose is to save the machine state of an object in order to be able to recreate it when needed.}}
\newglossaryentry{state database}{name={state database},description={A database stored off-chain, [i.e. on the computer of some user running an Ethereum client] which contains a trie structure mapping bytearrays (organized chunks of binary data) to other bytearrays. The \textsl{relationships} between each node on this trie constitutes a \textsc{mapping} of Ethereum's state.}}
\newglossaryentry{transaction}{name={transaction}, description={A piece of data, signed by an External Actor. It represents either a Message or a new Autonomous Object. Transactions are recorded into each block of the blockchain.}}
\newglossaryentry{Cryptographic hashing functions}{name={Cryptographic hashing functions}, description={Hash functions make secure blockchains possible by establishing universal inputs for which there are limited, usually only one, possible output yet that output is unique.}}
\newglossaryentry{state machine}{name={state machine}, description={The term \textsl{State Machine} is reserved for any simple or complex process that moves deterministically from one discrete state to the next.}}
\newglossaryentry{addresses}{name={addresses}, description={20 character strings, specifically the rightmost 20 characters of the \texttt{Keccak-256} hash of the RLP-derived mapping which contains the sender's address and the nonce of the block.}}
\newglossaryentry{EVM Code}{name={EVM Code}, description={The bytecode that the EVM can natively execute. Used to formally specify the meaning and ramifications of a message to an Account}}
\newglossaryentry{EVM Assembly}{name={EVM Assembly}, description={The human readable version of EVM code}}
\newglossaryentry{Storage State}{name={Storage State}, description={The information particular to a given account that is maintained between the times that the account's associated EVM Code runs}}
\newglossaryentry{Message}{name={Message}, description={Data (as a set of bytes) and Value (specified in Wei) that is passed between two accounts.}}
\newglossaryentry{Gas}{name={Gas}, description={The fundamental network cost unit; gas is paid for exclusively by Ether}}
\newglossaryentry{Contract}{name={Contract}, description={A piece of EVM Code that may be associated with an Account or an Autonomous Object}}
\newglossaryentry{Ethereum Runtime Environment}{name={Ethereum Runtime Environment}, description={The environment which is provided to an Autonomous Object executing in the EVM. Includes the EVM but also the structure of the world state on which the relies for certain I/O instructions including CALL \& CREATE}}
\newglossaryentry{beneficiary}{name={beneficiary}, description={The 20-character (160-bit)  address to which all fees collected from the successful mining of a block are transferred}}
\newglossaryentry{block header}{name={block header}, description={All the information in a block besides transaction information}}
\newglossaryentry{storage root}{name={storage root}, description={One aspect of an \textsc{account's state}: this is the hash of the trie\footnote{A particular path from root to leaf in the state database} that decides the \textsc{storage contents} of the account}}

\addbibresource{References.bib}


\setcounter{tocdepth}{3}
\setcounter{secnumdepth}{2}

%\makeatletter
%\patchcmd{\footnotetext}{\footnotesize}{\small\sffamily}{}{}
%\makeatother

\def\thesection{\arabic{section}}
%\renewcommand\thesubsection{\thesection\arabic{subsection}}


\setlength\bibitemsep{1.5\itemsep}
\newcommand{\sortitem}[1]{%
\DTLnewrow{list}% Create a new entry
\DTLnewdbentry{list}{description}{#1}% Add entry as description
}
\newenvironment{sortedlist}{%
\DTLifdbexists{list}{\DTLcleardb{list}}{\DTLnewdb{list}}% Create new/discard old list
}{%
\DTLsort{description}{list}% Sort list
\DTLforeach*{list}{\theDesc=description}{%
\item \theDesc}% Print each item
}

\newenvironment{alphafootnotes}
{\par\edef\savedfootnotenumber{\number\value{footnote}}
\renewcommand{\thefootnote}{\alph{footnote}}
\setcounter{footnote}{0}}
{\par\setcounter{footnote}{\savedfootnotenumber}}


\setlength{\columnseprule}{0pt}

\setlength{\columnsep}{5mm}

\hfuzz=0.84074pt

\setcounter{secnumdepth}{3}

\author{\large{\textbf{Micah Dameron}}}

\date{}

\title{\LARGE{Beigepaper: \\ An Ethereum Technical Specification}}

\makeindex
\begin{document}
	\setstretch{1.15}

	\begin{alphafootnotes}

	\pagecolor{beige}

	\maketitle

	\begin{center}\textbf{Abstract}\end{center}\par
		\abstract{The Ethereum Protocol is a deterministic but practically unbounded state-machine with two basic functions; the first being a globally accessible singleton state, and the second being a virtual machine that applies changes to that state. This paper explains the individual parts that make up these two factors.}
		\index{pseudocode}
		\index{abstract state-machine}
		\index{Yellowpaper}
		\index{deterministic}
		\index{singleton}
		\index{virtual machine}

	\begin{multicols*}{2}
	\TrickSupertabularIntoMulticols
	\begin{justify}
	
	
	\section{Imagining Bitcoin as a Computer}
		Ethereum utilizes the distributed ledger model that originated with Bitcoin and repurposes it to model a virtual computer, giving  machine level opcodes the same level of certainty as Bitcoin transactions. Just as sure as you can be certain that Bitcoin's ledger is accurate and that timestamps are correct through the Bitcoin \textit{consensus mechanism}, just so sure is it that machine instructions initiated on Ethereum \textit{will} execute.  	      
		\index{Bitcoin}
		\index{ledger}
		\index{opcodes}
		\index{certainty}
		\index{balance}
		\index{timestamped}
		\index{machine instructions}
		\index{ledger}
		
		
		In other words, programs executed on the Ethereum Blockchain are \textit{basically} unstoppable. However, that's not to say that writing bad code is impossible, because bad and buggy code can be written and executed on Ethereum the same way it can on any other computer. More precisely it means that code can be trusted to execute without any interference from \textit{external} non-network forces. This property arises from the inherent security of the blockchain which is built by, and maintained upon, cryptographic proofs.
		\index{unstoppable}
		\index{trusted}
		\index{executed}
		
		Just as in human systems power is allocated by hard work, by diligence, and by making oneself indispensable to others, the Ethereum network is subservient to others in terms of one thing only: \textsc{Ether}, the native currency for Ethereum. Everything which the system can do is bounded up in its ability to expend Ether in exchange for system performance. 

	\subsection{Native Currency}
		Because Ethereum aims not at being a currency, but at modeling a computer, there is a fundamental \textsl{network cost unit} used to mitigate the possibility of abusing the network with excessive computational expenditures. The smallest unit of currency in Ethereum is the Wei, which is equal to $\Xi10^{-18}$, where $\Xi$ stands for 1 Ether. All currency transactions in Ethereum, at the machine level, are counted in Wei. There is also the Szabo, which is $\Xi10^{-6}$,  and the Finney, which is $\Xi10^{-3}$. \\ 
		\index{native currency}
		\index{mining}
		\index{network cost unit}
		\index{Wei}
		\index{Szabo}
		\index{Finney}
		\index{Ether}
	\end{justify}
\raggedright
		\begin{tabular}{llr}
			\toprule
			\textbf{Unit} & \textbf{Ether} & \textbf{Wei} \\
			\midrule
			\scriptsize{Ether} & \scriptsize{$\Xi$\textbf{1}.000000000000000000} & \scriptsize{1,000,000,000,000,000,000} \\
			\scriptsize{Finney} & \scriptsize{$\Xi$0.00\textbf{1}000000000000000} & \scriptsize{1,000,000,000,000,000} \\
			\scriptsize{Szabo} & \scriptsize{$\Xi$0.00000\textbf{1}000000000000} & \scriptsize{1,000,000,000,000} \\
			\scriptsize{Wei} & \scriptsize{$\Xi$0.00000000000000000\textbf{1}} & \scriptsize{1} \\
			\bottomrule
		\end{tabular}
\justify
		\section{Memory and Storage}
				
				
			\subsection{World State}
				The \textit{World State} is divided by blocks; each new block represents a new world state. The world state is a mapping of \textbf{addresses} and \textbf{account states} through the use of the recursive length prefix standard (RLP). This information is stored as a  Merkle-Patricia tree in a \textsc{database backend}.\footnote{The database backend is accessed by users through an external application, most likely an Ethereum client; see also: \gls{state database}} that maintains a mapping of bytearrays to bytearrays.\footnote{A bytearray is specific set of bytes [data] that can be loaded into memory. It is a structure for storing binary data, e.g. the contents of a file.}\footnote{This permanent data structure makes it possible to easily recall any previous state with its root hash keeping the resources off-chain and minimizing on-chain storage needs.}As a whole, the state is the sum total of database relationships in the \textbf{ \gls{state database}}. 
			\index{world state}
			\index{mapping}
			\index{RLP}
			\index{account states}
			\index{account addresses}
			
			\subsubsection{Merkle-Patricia Trees} 
	
				
	Merkle-Patricia Trees are modified Merkletrees where nodes represent individual characters from hashes rather than each node representing an entire hash. This allows the state data structure itself to represent not only the intrinsically correct paths in the data, but also the requisite cryptographic proofs which go into making sure that a piece of data was valid in the first place. In other words, it keeps the blockchain valid by combining the structure of a standard Merkletree with the structure of a Radix Tree. Since all searching and sorting algorithms in Ethereum must be filtered through this stringently correct database, accuracy of information is guaranteed. \par		
		\index{merkletrees}
		\index{merkle-patricia trees}
		\index{merkle-patricia tries}
		\index{modified merkletrees}
		\index{tree database}
		\index{trie database}
		\index{data structure}
		
		The following is a search tree beginning with hexadecimal values \texttt{a} and \texttt{4}: \\
		
		
		\begin{forest}
		for tree={
			circle,
			black,
			draw,
			fill=blue!40,
		}
		[{}
		[{}, edge label={node [midway, left] {a}}
		[{}, edge label={node [midway, left] {2}}
		[{}, edge label={node [midway, left] {c}}
		[{}, edge label={node [midway, left] {7}}, label=below:a2c7
		]
		[,phantom]
		]
		[{}, edge label={node [midway, right] {b}}
		[{}, edge label={node [midway, left] {6}}, label=below:a2b6
		]
		[,phantom]
		]
		[,phantom]
		]
		[{}, edge label={node [midway, right] {6}}
		[,phantom]
		[{}, edge label={node [midway, right] {2}},	label=below:a62
		]
		]
		]
		[{}, edge label={node [midway, above left] {4}}
		[,phantom]
		[{}, edge label={node [midway, right] {9}}
		[,phantom]
		[{}, edge label={node [midway, right] {7}}
		[,phantom]
		[{}, edge label={node [midway, right] {f}}
		[,phantom]
		[{}, edge label={node [midway, right] {f}}, label=below:497ff
		]
		]
		]
		]
		[{}, edge label={node [midway, above right] {a}}
		[,phantom]
		[{}, edge label={node [midway, right] {c}}
		[,phantom]
		[{}, edge label={node [midway, right] {9}}
		[,phantom]
		[{}, edge label={node [midway, right] {5}}, label=below:4ac95
		[,phantom]
		]
		]
		]
		]
		]
		]
		\end{forest}

		\subsubsection{Tree Terminology\supercite{wiki:xxx}}
		\begin{enumerate}[label=\textbf{\alph*})]
			\item \textbf{Root Node} --  The top (first) node in a tree.
			\item \textbf{Child Node} --  A node directly connected to another node when moving away from the Root.
			\item \textbf{Parent Node} --  The converse notion of a child.
			\item \textbf{Sibling Nodes} --  A group of nodes with the same parent.
			\item \textbf{Descendant Node} --  A node reachable by repeated proceeding from parent to child.
			\item \textbf{Ancestor Node} --  A node reachable by repeated proceeding from child to parent.
			\item \textbf{Leaf Node} --  A node with no children.
			\item \textbf{Branch Node} --  A node with at least one child.
			\item \textbf{Degree} --  The number of subtrees of a node.
			\item \textbf{Edge} --  The connection between one node and another.
			\item \textbf{Path} --  A sequence of nodes and edges connecting a node with a descendant.
			\item \textbf{Level} --  The level of a node is defined by 1 + (the number of connections between the node and the root).
			\item \textbf{Node Height} --  The height of a node is the number of edges on the longest path between that node and a leaf.
			\item \textbf{Tree Height} --  The height of a tree is the height of its root node.
			\item \textbf{Depth} --  The depth of a node is the number of edges from the tree's root node to the node.
			\item \textbf{Forest} -- A forest is a set of n $\geq  0$ disjoint trees.
		\end{enumerate}
			\index{root node}
			\index{child node}
			\index{parent node}
			\index{sibling node}
			\index{descendant node}
			\index{ancestor node}
			\index{leaf node}
			\index{branch node}
			\index{tree degree}
			\index{tree edge}
			\index{tree path}
			\index{tree level}
			\index{node height}
			\index{tree height}
			\index{node depth}
			\index{forest}

			\subsubsection{Recursive Length Prefixes}
			\paragraph{Notation}: \texttt{rlp}
			\paragraph{Description}: RLP encodes arrays of nested binary data to an arbitrary depth; it is the main serialization method for data in Ethereum. RLP encodes mainly structure and does not pay heed to what type of data it is encoding. 
			\par
			\index{RLP}
			\index{nested binary data}
			\index{tree arbitrary depth}

		Positive RLP integers are represented with the most significant value stored at the lowest memory adddress (big endian)  and without any leading zeroes. As a result, the RLP integer value for \texttt{0} is represented by an empty byte-array. If a non-empty deserialized integer begins with leading zeros it is invalid.\supercite{EF2017}
			\par
			\index{RLP integers}
			\index{big endian}
			\index{no leading zeroes}
			\index{empty byte-array}
			\index{non empty deserialized integer}

			The global state database is encoded as RLP for fast traversal and inspection of data. In structure it constitutes a mapping between \textsl{addresses} and \textit{account states}. Since it is stored on node operator's computers, the tree can be traversed speedily and without network delay. RLP encodes values as byte-arrays, or as sequences of further values. \supercite{Wood2017} 
			\par 
			\index{global state database}
			\index{speedy traversal of data}
			\index{inspection  of data}
			\index{mapping between addresses}
			\index{mapping between account states}
			\index{node operator computer}
			\index{RLP encodes as byte-arrays}

			This means that:
			\\

			\texttt{%
				\begin{tabular}{ r l c l }
					if & rlp(x) &  = & bytearray \\
					then & rlp(bytearray) & = & true \\
					elif & rlp(x) & = & value \\
					then & rlp(value) & = & true \\
					elif & rlp(x) & = & null \\
					then & rlp(x) & = & false \\
				\end{tabular}
				}
				\\~\\

				\begin{enumerate}
							\item If the RLP-serialized byte-array contains a single byte integer value less than $128$, then the output is exactly equal to the input. 
				\end{enumerate}
				\index{RLP serialized byte-array}
				\index{single byte integer}


			\subsection{The Block}
				A block is made up of 17 different elements. The first 15 elements are part of what is called the \textsl{block header}.

				\index{block composition}
				\index{block header}

			\subsubsection{The Block Header}
				\paragraph{Description}: The information contained in a block besides the transactions list. This consists of:

				\begin{enumerate}
					\item \textbf{Parent Hash} -- This is the Keccak-256 hash of the parent block's header.
					\item \textbf{Ommers Hash} -- This is the Keccak-256 hash of the ommer's list portion of this block.
					\item \textbf{Beneficiary} -- This is the 20-byte address to which all block rewards are transferred.
					\item \textbf{State Root} -- This is the Keccak-256 hash of the root node of the state trie, after a block and its transactions are finalized.
					\index{parent hash}
					\index{ommers hash}
					\index{beneficiary}
					\index{state root}
					\index{keccak 256}
					\index{block rewards}
					
					\item \textbf{Transactions Root} -- This is the Keccak-256 hash of the root node of the trie structure populated with each transaction from a Block's transaction list.
					\item \textbf{Receipts Root} -- This is the Keccak-256 hash of the root node of the trie structure populated with the receipts of each transaction in the transactions list portion of the block.
					\item \textbf{Logs Bloom} -- This is the bloom filter composed from indexable information (log address and log topic) contained in the receipt for each transaction in the transactions list portion of a block.
					\item \textbf{Difficulty} -- This is the difficulty of this block -- a quantity calculated from the previous block's difficulty and its timestamp.
					\index{transactions root}
					\index{receipts root}
					\index{logs bloom}
					\index{difficulty}
					\index{keccak 256}

					\item \textbf{Number} -- This is a quantity equal to the number of ancestor blocks behind the current block.
					\item \textbf{Gas Limit} -- This is a quantity equal to the current maximum gas expenditure per block.
					\item \textbf{Gas Used} -- This is a quantity equal to the total gas used in transactions in this block.
					\item \textbf{Timestamp} -- This is a record of Unix's time at this block's inception.
					
					\index{number}
					\index{gas limit}
					\index{gas used}
					\index{timestamp}
					\index{keccak 256}

					\item \textbf{Extra Data} -- This byte-array of size 32 bytes or less contains extra data relevant to this block.
					\item \textbf{Mix Hash} -- This is a 32-byte hash that verifies a sufficient amount of computation has been done on this block.
					\item \textbf{Nonce} -- This is an 8-byte hash that verifies a sufficient amount of computation has been done on this block.
					\item \textbf{Ommer Block Headers} -- These are the same components listed above for any ommers.
				
					\index{extra data}
					\index{mix hash}
					\index{nonce}
					\index{ommer block headers}

					\end{enumerate}
				
					\subsubsection{Block Footer}
					\item \textbf{Transaction Series} -- This is the only non-header content in the block.
					
					\subsubsection{Block Number and Difficulty}
						Note that is the difficulty of the genesis block. The Homestead difficulty parameter, is used to affect a dynamic homeostasis of time between blocks, as the time between blocks varies, as discussed below, as implemented in EIP-2. In the Homestead release, the expo- nential difficulty symbol, causes the difficulty to slowly increase (every 100,000 blocks) at an exponential rate, and thus increasing the block time difference, and putting time pressure on transitioning to proof-of-stake. This effect, known as the “difficulty bomb”, or “ice age”, was explained in EIP-649 and delayed and implemented earlier in EIP-2, was also modified in EIP-100 with the use of x, the adjustment factor, and the denominator 9, in order to target the mean block time including uncle blocks. Finally, in the Byzantium release, with EIP-649, the ice age was delayed by creating a fake block number, which is obtained by substracting three million from the actual block number, which in other words reduced and the time difference between blocks, in order to allow more time to develop proof-of-stake and preventing the network from “freezing” up.\supercite{Wood2017}
					\index{block number}
					\index{genesis difficulty}
					\index{homestead difficulty parameter}
					\index{dynamic difficulty homeostasis}
					\index{exponential difficulty increase}
					\index{difficulty bomb}
					\index{byzantium}
					\index{ice age}
					\index{homestead}
					\index{EIP 2}
					\index{EIP 100}
					\index{EIP 649}
					
					\subsubsection{Account Creation} 

					Account creation definitively occurs with contract creation. Is related to: \texttt{init}. Lastly, there is the \texttt{body} which is the EVM-code that executes if/when the account containing it receives a message call.
					\index{account creation}
					\index{contract creation}
					\index{account init}
					\index{account body}

					\subsubsection{Account State}
				The account state contains details of any particular account during some specified world state. The account state is made up of four variables:
					\index{account state}
				\begin{enumerate}
					\item \textbf{nonce} The number of transactions sent from this address, or the number of contract creations made by the account associated with this address.
					\item \textbf{balance} The amount of \textbf{Wei} \textsc{owned} by this account. Stored as a key/value pair inside the state database. 
					\item \textbf{storage\_root} A 256-bit (32-byte) hash of the root node of a Merkle Patricia tree that encodes the storage contents of the account.\footnote{A particular path from root to leaf in the \textbf{\gls{state database}\index{state database}}} 
					\item \textbf{code\_hash} The hash of the EVM code of this account's contract. Code hashes are \textsc{stored} in the \textbf{\gls{state database}}. Code hashes are permanent and they are executed when the address belonging to that account \textsc{receives} a message call.

				\end{enumerate}
					\index{account nonce}
					\index{account balance}
					\index{account storage root}
					\index{account code hash}
					\index{account balance}
					\index{wei}
					\index{256 bit}
					\index{root node}
					\index{leaf node}
					\index{state database}

    					\subsubsection{Bloom Filter}
				The Bloom Filter is composed from indexable information (logger address and log topics) contained in each log entry from the receipt of each transaction in the transactions list. 

					\subsubsection{Transaction Receipts}


		\section{Processing and Computation}
	
					\subsection{The Transaction}
	
			The basic method for Ethereum accounts to interact with eachother. The transaction is a single cryptographically signed instruction sent to the Ethereum network. There are two types of transactions: \textsc{message calls} and \textsc{contract creations}. Transactions lie at the heart of Ethereum, and are entirely responsible for the dynamism and flexibility of the platform. Transactions are the bread and butter of state transitions, that is of block additions, which contain all of the computation performed in one block. Each transaction applies the execution changes to the \textit{machine state}, a temporary state which consists of all the required changes in computation that must be made before a block is finalized and added to the world state.
					\index{transaction}
					\index{state transition}
					\index{machine state}
		\subsubsection{Transactions Root}

			\paragraph{Notation}: \texttt{listhash}
			\paragraph{Alternatively:} Transactions Root
			\paragraph{Description}: The \texttt{Keccak-256} hash of the root node that precedes the \texttt{transactions} in the \texttt{transactions\_list} section of a Block.			
			
			\begin{enumerate}
				
				\item \textbf{Nonce} -- The number of transactions sent by the sender.
				\item \textbf{Gas Price} -- The number of Wei to pay the network for unit of gas.
				\item \textbf{Gas Limit} -- The maximum amount of gas to be used in while executing a transaction.			
				\item \textbf{To} -- The 20-character recipient of a message call.\footnote{In the case of a contract creation this is 0x000000000000000000.}
				\item \textbf{Value} The number of Wei to be transferred to the recipient of a message call.\footnote{In the case of a contract creation, an endowment to the newly created contract account.}
				\item \textbf{v, r, s} 
			\end{enumerate}
			
		\subsection{State Transition Function}
		State Transitions come about through the State Transition Function; this is a high-level abstraction of several operations in Ethereum which comprise the overall act of taking changes from  the \textit{machine state} and adding  them to the world state.
					\index{gas price}
					\index{gas limit}
					\index{to}
					\index{value}
	
		\subsection{Mining}
			The \texttt{Block Beneficiary} is the 160-bit (20-byte) address to which all fees collected from the successful mining of a block are transferred. \texttt{Apply Rewards} is the third process in \texttt{block\_finalization} that sends the mining reward to an account's address. This is a scalar value corresponding to the difficulty level of a current block. 

		\subsection{Verification}
			The process in The EVM that verifies Ommer Headers

					\index{EVM}
					\index{verification}
					\index{160 bit}
					\index{apply rewards}
					\index{block beneficiary}
					\index{block reward}

		\subsection{Sender Function} A description that maps transactions to their sender using \texttt{ECDSA} of the SECP-256k1 curve,

		\subsection{Serialization/Deserialization}
			This function expands a positive-integer value to a big-endian byte-array of minimal length. When accompanied by a $\cdot$ operator, it signals sequence concatenation.  The \texttt{big\_endian} function  accompanies RLP serialization and deserialization.
					\index{serialization}
					\index{deserialization}
		\subsection{Ethereum Virtual Machine}
			The EVM has a simple stack-based architecture. The word size of the machine and thus size of stack is 256-bit. This was chosen to facilitate the Keccak-256 hash scheme and elliptic-curve based computation. The memory model is a simple word-addressed byte-array. The memory stack has a maximum size of 1024-bits. The machine also has an independent storage model; this is similar in concept to the memory but rather than a byte array, it is a word-addressable word array. Unlike memory, which is volatile, storage is non-volatile and is maintained as part of the system state. 
					\index{stack based architecture}
					\index{stack based}
					\index{word size}
					\index{256 bit}
					\index{memory}
					\index{memory size}
					\index{keccak 256}
					\index{hash scheme}
					\index{elliptic curve cryptography}
					\index{elliptic curve }
					\index{elliptic curve computation}
					\index{memory model}
					\index{word addressed}
					\index{byte array}
					\index{memory stack}
					\index{machine storage}
					\index{storage model}
					\index{word array}
					\index{word addressable}
					\index{memory model volatility of}
					\index{system state}

			All locations in both storage and memory are well-defined initially as zero. The machine does not follow the standard von Neumann architecture. Rather than storing program code in generally-accessible memory or storage, it is stored separately in a virtual ROM interactable only through specialized instructions. 
			
					\index{well defined storage}
					\index{well defined memory}
					\index{non-standard architecture}
					\index{virtual ROM}

			The machine can have exceptional execution for several reasons, including stack underflows and invalid instructions. Like the out-of-gas exception, they do not leave state changes intact. Rather, the machine halts immediately and reports the issue to the execution agent (either the transaction processor or, recursively, the spawning execution environment) which will deal with it separately. 
			
					\index{exceptional halt}
					\index{stack underflow}
					\index{invalid instruction}
					\index{out-of-gas}
					\index{state unchanged}
					\index{machine halt}
					\index{report exception}
			
			\subsubsection{Fees} Fees (denominated in gas) are charged under \textsc{three} distinct circumstances, all three as prerequisite to the execution of an operation.\supercite{Wood2017} The \textbf{first} and most common is \textsl{the fee intrinsic to the computation of the operation}. \textbf{Secondly}, gas may be deducted \textsl{in order to form the payment for a subordinate message call or contract creation}; this forms part of the payment for the CREATE, CALL and CALLCODE operations. \textbf{Finally}, \textsl{gas may be paid due to an increase in the usage of the memory.} 
					\index{fees}
					\index{gas}
					\index{execution}
					\index{computation of operation}
					\index{gas deducted}
					\index{payment}
					\index{message call}
					\index{gas paid for increased use of memory}


			Over an account’s execution, the total fee for memory-usage payable is proportional to smallest multiple of 32 bytes that are required such that all memory indices (whether for read or write) are included in the range. This is paid for on a just-in-time basis; as such, referencing an area of memory at least 32 bytes greater than any previously indexed memory will certainly result in an additional memory usage fee. Due to this fee it is highly unlikely that addresses will trend above 32-bit bounds.\supercite{Wood2017} 
					\index{total fee}
					\index{memory usage fee}
					
			Implementations must be able to manage this eventuality. Storage fees have a slightly nuanced behaviour to incentivize minimization of the use of storage (which corresponds directly to a larger state database on all nodes), the execution fee for an operation that clears an entry in the storage is not only waived, a qualified refund is given; in fact, this refund is effectively paid up-front since the initial usage of a storage location costs substantially more than normal usage. \supercite{Wood2017}
					\index{gas price}
					\index{minimize storage use}
					\index{gas refund clearing space}
					
			\subsection{Execution}
				The execution of a transaction defines the state transition function: \texttt{stf}. However, before any transaction can be executed it needs to go through the initial tests of intrinsic validity.
				\index{intrinsic validity}
			\subsubsection{Intrinsic Validity}
				The criteria for intrinsic validity are as follows:
				\begin{itemize}
				\item The transaction follows the rules for \textsl{well-formed RLP} (recursive length prefix.)
				\item The \textsl{signature} on the transaction is valid.
				\item The \textsl{nonce} on the transaction is valid, i.e. it is equivalent to the sender account's current nonce.
				\item The \texttt{gas\_limit} is greater than or equal to the \texttt{intrinsic\_gas} used by the transaction.
				\item The sender's account balance contains the cost required in up-front payment.
				\end{itemize}
					\index{well-formed RLP}
					\index{transaction signature}
					\index{transaction nonce}
					\index{gas limit}
					\index{sender account}
					\index{upfront payment}

			\subsubsection{Transaction Receipt}
				While the amount of gas used in the execution and the accrued log items belonging to the transaction are stored, information concering the result of a transaction's execution is stored in the transaction receipt \texttt{tx\_receipt}. The set of log events which are created through the execution of the transaction, \texttt{logs\_set} in addition to the bloom filter which contains the actual information from those log events \texttt{logs\_bloom} are located in the transaction receipt. In addition, the post-transaction state \texttt{post\_transaction(state)} and the amount of gas used in the block containing the transaction receipt post(gas\_used) are stored in the transaction receipt. As a result, the transaction receipt is a record of any given \texttt{execution}.
					\index{gas used}
					\index{transaction receipt}
					\index{log items}
					\index{log events}
					\index{transaction execution}
					\index{post transaction state}
					
					A valid transaction execution begins with a permanent change to the state: the nonce of the sender account is increased by one and the balance is decreased by the \texttt{collateral\_gas}\footnote{Designated ``\texttt{intrinsic\_gas}'' in the Yellowpaper} which is the amount of gas a transaction is required to pay prior to its execution. The original transactor will differ from the sender if the message call or contract creation comes from a contract account executing code. 
			
			After a transaction is executed, there comes a \textsc{provisional state}, which is a pre-final state. Gas used for the execution of individual EVM opcodes prior to their potential addition to the \texttt{world\_state} creates:
			
		\begin{itemize}
			\item provisional state
			\item \texttt{intrinsic gas}, and
			\item an associated substate  
		\end{itemize}
		
		\begin{itemize}
        		\item The accounts tagged for self-destruction following the transaction's completion. \texttt{self\_destruct(accounts)}
        		\item The \texttt{logs\_series}, which creates checkpoints in EVM code execution for frontend applications to explore, and is made up of the\texttt{logs\_set} and \texttt{logs\_bloom} from the \texttt{tx\_receipt}.
			\item The refund balance.\footnote{The \textsc{sstore} operation increases the amount refunded by resetting contract storage to zero from some non-zero state.}
		\end{itemize}

		Code execution always depletes \texttt{gas}. If gas runs out, an out-of-gas error is signaled (\texttt{oog}) and the resulting state defines itself as an empty set; it has no effeffect on the world state. This describes the transactional nature of Ethereum. In order to affect the \textsc{world state}, a transaction must go through completely or not at all. 

				\subsubsection{Code Deposit}
				If the initialization code completes successfully, a final contract-creation cost is paid, the code-deposit cost, \textsc{c}, proportional to the size of the created contract's code. 
	
				\subsubsection{Execution Model}
				\paragraph{Basics}: The stack-based \textsl{virtual machine} which lies at the heart of the Ethereum and performs the actions of a computer. This is actually an instantial runtime that executes several substates, as EVM computation instances, before adding the finished result, all calculations having been completed, to the final state  via the \texttt{finalization function}. 
	
				In addition to the \texttt{system state} and the \texttt{remaining gas} for computation there are several pieces of important information used in the execution environment that the execution agent must provide:
					\index{system state}
					\index{remaining gas}
					\index{stack based}
					\index{virtual machine}
					\index{instantial runtime}
					\index{substates}
					\index{evm computation instances}
					\index{finalization function}
					\index{execution environment}

				\begin{itemize}
					\item \texttt{account\_address}, the address of the account which owns the code that is executing.
					\item \texttt{sender\_address} the sender address of the transaction that originated this execution.
					\item \texttt{originator\_price} the price of gas in the transaction that originated this execution.
					\item \texttt{input\_data}, a byte array that is the input data to this execution; if the execution agent is a transaction, this would be the transaction data.
					\item \texttt{account\_address}  the address of the account which caused the code to be executing; if the execution agent is a transaction, this would be the transaction sender.
					\item \texttt{newstate\_value} the value, in Wei, passed to this account if the execution agent is a transaction, this would be the transaction value.\supercite{Wood2017}
					\item \texttt{code array} the byte array that is the machine code to be executed.\supercite{Wood2017}
					\item \texttt{block\_header} the block header of the present block.
					\item \texttt{stack\_depth} the depth of the present message-call or contract-creation (i.e. the number of {\small CALL}s or {\small CREATE}s being executed at present).\supercite{Wood2017}
				\end{itemize}
					\index{account address}
					\index{sender address}
					\index{originator price}
					\index{input data}
					\index{account address}
					\index{newstate value}
					\index{code array}
					\index{block header}

				The execution model defines the \texttt{state\_transition function}, which can compute the \texttt{resultant state}, the \texttt{remaining\_gas}, the \texttt{accrued\_substate} and the \texttt{resultant\_output}, given these definitions. For the present context, we will define it where the accrued substate is defined as the tuple of the \texttt{self-destructs\_set}, the \texttt{log\_series}, the \texttt{touched\_accounts} and the \texttt{refunds}.\supercite{Wood2017}
					\index{execution model}
					\index{state transition function}
					\index{resultant state}
					\index{remaining gas}
					\index{accrued substate}
					\index{resultant output}
					\index{self-destructs set}
					\index{log series}
					\index{touched accounts}
					\index{refunds}

				\subsubsection{Execution Overview} The \texttt{execution\_function}, in most practical implementations, \textbf{will be modeled as an iterative progression of the pair comprising the full \texttt{system\_state} and the \texttt{machine\_state}.} It's defined recursively with the \texttt{iterator\_function}, which defines the result of a single cycle of the state machine, together with the \texttt{halting\_check} function, which determines if the present state is an exceptional halting state of the machine and \texttt{output\_data} of the instruction if the present state is a \texttt{controlled\_halt} of the machine. An empty sequence/series indicates that execution should halt, while the empty set indicates that execution should continue.
					\index{execution function}
					\index{iterative progression}
					\index{iterator function}
					\index{state machine cycle}
					\index{halting function}
					\index{halting state}


				When evaluating execution, we extract the remaining gas from the resultant machine state. It is thus cycled (recursively or with an iterative loop) until either \texttt{exceptional\_halt} becomes true indicating that the present state is exceptional and that the machine must be halted and any changes discarded or until H becomes a series (rather than the empty set) indicating that the machine has reached a controlled halt. 
					\index{extract remaining gas}
					\index{exceptional halt}
					
				The machine state  is defined as the tuple which are the \textbf{gas available}, \textbf{the program counter}, \textbf{the memory contents}, \textbf{the active number of words in memory} (counting continuously from position 0), \textbf{and the stack contents}. The memory contents are a series of zeroes of size $2^{256}$.\supercite{Wood2017}
				
				\subsubsection{The Execution Cycle} Stack items are added or removed from the left-most, lower-indexed portion of the series; all other items remain unchanged:  The gas is reduced by the instruction’s gas cost and for most instructions, the program counter increments on each cycle, for the three exceptions, we assume a function J, subscripted by one of two instructions, which evaluates to the according value: otherwise In general, we assume the memory, self-destruct set and system state don’t change: However, instructions do typically alter one or several components of these values. 
				
				\paragraph{Provisional State}
				A smaller, temporary state that is generated during transaction execution. It contains three sets of data:

			\subsubsection{Message Calls}
			A message call can come from a transaction or internally from contract code execution. It contains the field \textsc{data}, which consists of user data input to a message call. Messages allow communication between accounts (whether contract or external.) Messages can come in the form of \texttt{msg\_calls} which give output data. If it is a contract account, this code gets executed when the account receives a message call. Message calls and contract creations are both \textsl{transactions}, but contract creations are never considered the same as message calls. Message calls always transfer some amount of value to an account. If the message call is an account creation transaction then the value given is takes on the role of an endowment toward the new account. Every time an account receives a message call it returns the \texttt{body}, something which is triggered by the \texttt{init} function. User data input to a \texttt{message\_call}, structured as an unlimited size byte-array.
					\index{init}
					\index{message call}

\paragraph{Universal Gas}Message calls always have a universally agreed-upon cost in gas. There is a strong distinction between contract creation transactions and message call transactions. Computation performed, whether it is a contract creation or a message call, represents the currently legal valid state. There can be no invalid transactions from this point. \supercite{Wood2017} There is also a message call/contract creation \textit{stack}. This stack has a depth, depending on how many transactions are in it. Contract creations and message calls have entirely different ways of executing, and are entirely different in their roles in Ethereum. The concepts can be conflated. Message calls can result in computation that occurs in the next state rather than the current one. If an account that is currently executing receives a message call, no code will execute, because the account might exist but has no code in it yet. To execute a message call transactions are required:
					\index{universal gas}
					\index{contract creation transactions}
					\index{message call transactions}
					\index{computation}
					\index{valid state}

\begin{itemize}
	\item \texttt{sender}
	\item \texttt{transaction originator} 
	\item \texttt{recipient}
	\item \texttt{account} (usually the same as the recipient) 
	\item \texttt{available gas} 
	\item \texttt{value}
	\item \texttt{gas price}
	\item An arbitrary length byte-array. \texttt{arb array}
	\item \texttt{present depth} of the message call/contract creation stack.
\end{itemize}

					\index{sender}
					\index{transaction originator}
					\index{recipient}
					\index{account}
					\index{available gas}
					\index{value}
					\index{gas price}
					\index{arbitrary length byte-array}
					\index{present depth}
					\index{contract creation stack}
					\index{message call}

				\subsubsection{Contract Creation}
				When \textsc{init} is executed it returns the \textsc{body}. Init is executed only once at \textsc{account\_creation}, and permanently discarded after that. Contract creation transactions are equal the recursive length prefix of an empty byte-sequence. 		
					\index{contract creation}
					\index{init}
					\index{body}
					\index{account creation}
					\index{empty byte-sequence}


				\subsubsection{Execution Environment}
					 The Ethereum Runtime Environment is the environment under which Autonomous Objects execute in the EVM: the EVM runs as a part of this environment.  

					\index{ere}
					\index{ethereum runtime environment}
					\index{autonomous objects}

				\subsubsection{Big Endian Function} This function expands a positive-integer value to a big-endian byte array of minimal length. When accompanied by a $\cdot$ operator, it signals sequence concatenation. The \texttt{big\_endian} function  accompanies RLP serialization and deserialization.
					\index{big endian function}
					\index{positive integer}
					\index{sequence concatenation}
					\index{rlp}
					\index{serialization}
					\index{deserialization}


				\subsection{Gas}
				Gas is the fundamental network cost unit converted to and from Ether as needed to complete the transaction while it is sent. Gas is arbitrarily determined at the moment it is needed, by the block and according to the miners decision to charge certain fees. Miners choose which gas prices they want to accept.
					\index{miner choice}
					\index{arbitrarily determined}
					\index{gas}
					\index{network cost unit}
	
				\subsubsection{Gas Price/Gas Limit}
					\texttt{Gas price} is a value equal to the current limit of gas expenditure per block, according to the miners. Any unused gas is refunded to the sender. The canonical gas limit of a block is expressed  and is stabilized by the \texttt{time\_stamp} of the block.
					\index{value}
					\index{gas expenditure per block}
					\index{miners}
					\index{unused gas}
					\index{refunded}
					\index{canonical gas}
					\index{time stamp}
					\index{block}

				\paragraph{Gas Price Stability}
					Where \texttt{new\_header} is the new block’s header, but without the nonce and mix-hash components, d being the current DAG, a large data set needed to compute the mix-hash, and PoW is the proof-of-work function this evaluates to an array with the first item being the mix-hash, to proof that a correct DAG has been used, and the second item being a pseudo-random number cryptographically dependent on it. Given an approximately uniform distribution in the range the expected time to find a solution is proportional to the difficulty.\supercite{Wood2017} 
					
					\index{DAG}
					\index{mix hash}
					\index{correct DAG}
					\index{difficulty}
					
					This is the foundation of the security of the blockchain and is the fundamental reason why a malicious node cannot propagate newly created blocks that would otherwise overwrite (“rewrite”) history. Because the nonce must satisfy this requirement, and because its satisfaction depends on the contents of the block and in turn its composed transactions, creating new, valid, blocks is difficult and, over time, requires approximately the total compute power of the trustworthy portion of the mining peers. Thus we are able to define the block header validity function.
				
					\index{block header validity function}
					\index{nonce}
					\index{block contents}
				
				\paragraph{Gasused}
				A value equal to the total gas used in transactions in this block. 
					\index{gas used}

				\subsubsection{Machine State}
				The machine state is a tuple consisting of five elements:
		
		\begin{enumerate}
			\item \texttt{gas\_available}
			\item \texttt{program\_counter}
			\item \texttt{memory\_contents} A series of zeroes of size $2^{256}$
			\item \texttt{memory\_words.count}
			\item \texttt{stack\_contents}
		\end{enumerate}
					\index{machine state}
					\index{gas available}
					\index{program counter}
					\index{memory contents}
					\index{memory word count}
					\index{stack contents}

		There is also, \texttt[to\_execute]: the current operation to be executed
					\index{to execute}
		
				\subsubsection{Exceptional Halting}
				An exceptional halt may be caused by four conditions existing on the stack with regard to the next opcode in line for execution:
		
					\begin{verbatim}
					if 
					out_of_gas = true 
					or
					bad_instruction = true
					or
					bad_stack_size = true
					or
					bad_jumpdest = true
					then throw exception
					else exec opcode x
					then init control_halt
					\end{verbatim}	
		
		Exceptional halts are reserved for opcodes that fail to execute. They can never be caused through an opcode's actual execution.
		
		\begin{itemize}
			\item The amount of remaining gas in each transaction is extracted from information contained in the \texttt{machine\_state} 
			\item A simple iterative recursive  loop\supercite{Wood2017} with a boolean  value: 
		\begin{itemize}
				\item\textbf{true} indicating that in the run of computation, an exception was signaled
				\item\textbf{false} indicating in the run of computation, no exceptions were signaled. If this value remains false for the duration of the execution until the set of transactions becomes a series (rather than an empty set.) This means that the machine has reached a controlled halt. 
			\end{itemize}
		\end{itemize}
    					\index{controlled halt}
					\index{transaction series}
					\index{empty set}

				\paragraph{Substate}
				
				 A smaller, temporary state that is generated during transaction execution and runs parallel to \texttt{machine state}. It contains three sets of data:
				
				\begin{itemize}
					\item The accounts tagged for self-destruction following the transaction's completion. \texttt{self\_destruct(accounts)}
					\item The \texttt{logs\_series}, which creates checkpoints in EVM code execution for frontend applications to explore, and is made up of the\texttt{logs\_set} and \texttt{l
						ogs\_bloom} from the \texttt{tx\_receipt}.
					\item The refund balance.\footnote{The \textsc{sstore} operation increases the amount refunded by resetting contract storage to zero from some non-zero state.}
				\end{itemize}
				
					\index{tagged for self destruction}
					\index{logs series}
					\index{checkpoints}
					\index{logs set}
					\index{logs bloom}
					\index{transaction receipt}

			\subsubsection{EVM Code}
				The bytecode that the EVM can natively execute. Used to explicitly specify the meaning of a message to an account. A \texttt{contract} is a piece of EVM Code that may be associated with an Account or an Autonomous Object. \textbf{EVM Assembly} is the human readable version of EVM Code.
					\index{EVM code}
					\index{natively execute}
					\index{EVM assembly}
					\index{explicitly specify meaning}
\subsection{Blocktree to Blockchain}
		
		The canonical blockchain is a path from root to leaf through the entire block tree. In order to have consensus over which path it is, conceptually we identify the path that has had the most computation done upon it, or, the heaviest path. Clearly one factor that helps determine the heaviest path is the block number of the leaf, equivalent to the number of blocks, not counting the unmined genesis block, in the path. The longer the path, the greater the total mining effort that must have been done in order to arrive at the leaf. This is akin to existing schemes, such as that employed in Bitcoin-derived protocols. Since a block header includes the difficulty, the header alone is enough to validate the computation done. Any block contributes toward the total computation or total difficulty of a chain. Thus we define the total difficulty of \texttt{this\_block} recursively by the difficulty of its parent block and the block itself.
					\index{canonical blockchain}
					\index{heaviest path}
					\index{block number}
					\index{genesis block}
					\index{mining effort}
					\index{totaly difficulty}
		The jobs of miners and validators are as follows: \texttt{Validate (or, if mining, determine) ommers; validate (or, if mining, determine) transactions; apply rewards; verify (or, if mining, compute a valid) state and nonce.} 
					\index{compute valid state}
					\index{compute valid nonce}

		\subsection{Ommer Validation} The validation of ommer headers means nothing more than verifying that each ommer header is both a valid header and satisfies the relation of Nth-generation ommer to the present block. The maximum of ommer headers is two. 
					\index{ommer validation}
					\index{ommer headers}
					\index{valid header}
		\subsection{Transaction Validation} The given gasUsed must correspond faithfully to the transactions listed, the total gas used in the block, must be equal to the accumulated gas used according to the final transaction.
					\index{transaction validation}
					\index{gas used}
					\index{transactions}
					\index{total gas used}
					\index{accumulated gas used}
		\subsection{Reward Application} The application of rewards to a block involves raising the balance of the accounts of the beneficiary address of the block and each ommer by a certain amount. We raise the block’s beneficiary account; for each ommer, we raise the block’s beneficiary by 1 an additional 32 of the block reward and the beneficiary of the ommer gets rewarded depending on the block number. This constitutes the \texttt{block\_finalization state\_transition\_function} If there are collisions of the beneficiary addresses between ommers and the block two ommers with the same beneficiary address or an ommer with the same beneficiary address as the present block,
					\index{apply rewards}
					\index{beneficiary address}
					\index{ommer}
					\index{block reward}
					\index{block number}
					\index{block finalization state transition function}
					\index{collisions}

			additions are applied cumulatively. We define the block reward as 3 Ether: State \& Nonce Validation. We may now define the function, that maps a block B to its initiation state: otherwise Here, that means the hash of the root node of a trie of state x; it is assumed that implementations will store this in the state database, trivial and efficient since the trie is by nature a resilient data structure. And finally define the \texttt{block\_transition\_function}, which maps an incomplete block to a complete block with a specified dataset. As specified at the beginning of the present work, the \texttt{state\_transition\_function}, which is defined in terms of, the \texttt{block\_finalisation\_function} and, the \texttt{transaction\_evaluation\_function}. As previously detailed, there is the nth corresponding status code, logs and cumulative gas used after each transaction, the fourth component in the tuple, has already been defined in terms of the logs). 
					\index{nonce validation}
					\index{incomplete block}
					\index{complete block}
					\index{status code}
					\index{cumulative gas}
		\paragraph{}The nth state is given from applying the corresponding transaction to the state resulting from the previous transaction (or the block’s initial state in the case of the first BYZANTIUM VERSION 3475aa8 - 2018-01-26 14 such transaction): otherwise In certain cases we take a similar approach defining each item as the gas used in evaluating the corresponding transaction summed with the previous item (or zero, if it is the first), giving us a running total: the function is used that was defined in the transaction execution function. We define R[n] a similar manner. Finally, we define \texttt{new state} given the \texttt{block reward function} applied to the final transaction’s \texttt{resultant state}, thus the complete block-transition mechanism, less PoW, the proof-of-work function is defined. 
					\index{transaction execution function}
					\index{block reward function}
		
		
		\subsection{Mining Proof-of-Work} Proof that a certain amount of mining has been done exists as a cryptographic probability statement which asserts beyond reasonable doubt that a particular amount of computation has been expended in the determination of some token value n. It is utilised to enforce the uncompromisable security of the blockchain. Since mining blocks comes with an attached reward, the proof-of-work not only functions as a method of securing confidence in the future and past state of the machine, but also as a wealth distribution mechanism. The proof of work function should be as accessible as possible to as many people as possible. 
					\index{proof-of-work}
					\index{probability statement}

		To make the Ethereum Blockchain ASIC resistant, the Proof-of-Work mechanism has been designed to be  sequential memory-hard. This means that the nonce requires a lot of memory and bandwidth such that the memory cannot be used in parallel to discover multiple nonces simultaneously. Therefore, the proof-of-work function takes the form of $2^256m = Hm$ is the new block’s header but without the nonce and mix-hash components; $Hn$ is the nonce of the header; $d$ is a large data set needed to compute the mix hash and $H d$ is the new block’s difficulty value. PoW is the proof-of-work function which evaluates to an array with the first item being the mix hash and the second item being a pseudorandom number which is cryptographically dependent on $H$ and $d$. The name for this algorithm is Ethash. 
		\subsubsection{Ethash}	\index{asic resistant}
					\index{ethash}
					\index{mix hash}
		Ethash is the PoW algorithm for Ethereum 1.0. It is the latest version of Dagger-Hashimoto, introduced by Vitalik Buterin. The general route that the algorithm takes is as follows: There exists a seed which can be computed for each block by scanning through the block headers up until that point. From the seed, one can compute a pseudorandom cache, J cacheinit bytes in initial size. Light clients store the cache. From the cache, we can generate a dataset, \texttt{ds} bytes in initial size, with the property that each item in the dataset depends on only a small number of items from the cache. Full clients and miners store the dataset. The dataset grows linearly with time. Mining involves grabbing random slices of the dataset and hashing them together. Verification can be done with low memory by using the cache to regenerate the specific pieces of the dataset that you need, so you only need to store the cache. The large dataset is updated once every 1 epoch blocks, so the vast majority of a miner’s effort will be reading the dataset, not making changes to it. The mentioned parameters as well as the algorithm is explained in detail in appendix J. 12. Implementing Contracts There are several patterns of contracts engineering that allow particular useful behaviours; two of these that I will briefly discuss are data feeds and random numbers. 
					\index{seed}
					\index{cache}
					\index{dataset}
		\subsubsection{Difficulty Mechanism}
			This mechanism enforces a homeostasis in terms of the time between blocks; a smaller period between the last two blocks results in an increase in the difficulty level and thus additional computation required, lengthening the likely next period. Conversely, if the period is too large, the difficulty, and expected time to the next block, is reduced. The \textit{Total Computation} is the difficulty state of the entire Ethereum blockchain. The \textit{Block Difficulty} is not a state of the blockchain, but is local--particular to each specific block. You reach the total difficulty by summing the difficulty of all previous blocks and then adding the present one.  \par
			\index{total difficulty}	
			\index{block difficulty}
			\index{difficulty mechanism}
			\index{cumulative difficulty}
			The \textbf{GHOST Protocol} provides an alternative solution to double-spend attacks from the original solution in Satoshi Nakamoto's Bitcoin Whitepaper. Nakamoto solved the problem of double-spending by requiring the network to agree on the first valid block.  impossible to submit a ``double-spend''  block without  having at least 50\% of the network's mining power to force the longest chain. This is because the network automatically chooses the longest chain. So even if one wanted to  submit two spend transactions in a row, the network simply picks whichever one comes first, ignoring the second because it no longer pertains to the longest chain (which now contains the first block that was sent) so the would-be hacker needs to submit a new block, as the first double block is no longer feasible.\par
			\index{GHOST protocol}
			\index{double-spend problem}
			\index{longest chain}
			\index{50\% attack}
			\index{Satoshi Nakamoto}
			\index{Bitcoin Whitepaper}
					\index{dataset slice}
		\subsection{Pseudorandom Numbers} Pseeudo-random numbers be generated by utilizing data which is generally unknowable at the time of transacting. Such data might include the block’s hash, the block’s timestamp or the block’s beneficiary address. The BLOCKHASH opcode uses the previous 256 blocks as pseudo-random numbers. One could automate this randomness by adding a fixed value and hashing the result.   
					\index{pseudorandom number generation}
					\index{BLOCKHASH}
		\subsection{Chainsize Limits} The state database won’t be forced to maintain all past state trie structures into the future. It should maintain an age for each node and eventually discard nodes that are neither recent enough nor checkpoints; checkpoints, or a set of nodes in the database that allow a particular block’s state trie to be traversed, could be used to place a maximum limit on the amount of computation needed in order to retrieve any state throughout the blockchain. Blockchain consolidation could be used in order to reduce the amount of blocks a client would need to download to act as a full mining, node. A compressed archive of the trie structure at given points in time (perhaps one in every 10,000th block) could be maintained by the peer network, effectively recasting the genesis block. This would reduce the amount to be downloaded to a single archive plus a hard maximum limit of blocks. Finally, blockchain compression could perhaps be con- ducted: nodes in state trie that haven’t sent/received a transaction in some constant amount of blocks could be thrown out, reducing both Ether-leakage and the growth of the state database.\supercite{Wood2017}
					\index{state database}
					\index{age}
					\index{discard nodes}
					\index{checkpoint nodes}
		\subsection{Scalability} Scalability is a constant concern. With a generalized state transition function, it becomes difficult to partition space, but several strategies exist that may provide for scaling.
					\index{scalability}
		\subsubsection{Sharding}
		Parallelization of transaction combination and block building.
					\index{sharding}
		\subsubsection{Casper}
					\index{casper}
		\subsubsection{Plasma}
					\index{plasma}
		
			
	\end{multicols*}

	\appendix

	\section{Opcodes}

		\subsection{0x10's: Comparisons and Bitwise Logic Operations}
			\begin{longtable}{|cP{2.8cm}cccp{6cm}|}
			\hline
			\textbf{Data} & \textbf{Opcode} & \textbf{Gas} & \textbf{Input} & \textbf{Output} & \textbf{Description} \\
			\hline
			0x00 & STOP & 0 & 0 & 0 & Halts execution. \\
			0x01 & ADD & 3 & 2 & 1 & Addition operation. \\
			0x02 & MUL & 5 & 2 & 1 & Multiplication operation. \\
			0x03 & SUB & 3 & 2 & 1 & Subtraction operation. \\
			0x04 & DIV & 5 & 2 & 1 & Integer division operation. \\
			0x05 & SDIV & 5 & 2 & 1 & Signed integer division operation (truncated.)\\
			0x06 & MOD & 5 & 2 & 1 & Modulo remainder operation. \\
			0x07 & SMOD & 5 & 2 & 1 & Signed modulo remainder operation. \\
			0x08 & ADDMOD & 8 & 3 & 1 & Modulo addition operation. \\
			0x09 & MULMOD & 8 & 3 & 1 & Modulo multiplication operation. \\
			0x0a & EXP & 10 & 2 & 1 & Exponential operation. \\
			0x0b & SIGNEXTEND & 5  & 2 & 1 & Extend the length of two's complementary signed integer. \\
			0x10 & LT & 3 & 2 & 1 & Less-than comparison. \\
			0x11 & GT & 3 & 2 & 1 & Greater-than comparison. \\
			0x12 & SLT & 3 & 2 & 1 & Signed less-than comparison. \\
			0x13 & SGT & 3 & 2 & 1 & Signed greater-than comparison. \\
			0x14 & EQ & 3 & 2 & 1 & Equality comparison. \\
			0x15 & ISZERO & 3 & 1 & 1 & Simple not operator. \\
			0x16 & AND & 3 & 2 & 1 & Bitwise \textsc{and} operation. \\
			0x17 & OR & 3 & 2 & 1 & Bitwise \textsc{or} operation. \\
			0x18 & XOR & 3 & 2 & 1 & Bitwise \textsc{xor} operation. \\
			0x19 & NOT & 3 & 1 & 1 & Bitwise \textsc{not} operation. \\
			0x1a & BYTE & 3 & 2 & 1 & Retrieve single byte from word. \\
			\hline
			\end{longtable}

	        \subsection{0x20's: SHA3}
			\begin{longtable}{|cP{2.8cm}cccp{6cm}|}
			\hline
		        \textbf{Data} & \textbf{Opcode} & \textbf{Gas}  & \textbf{Input}  & \textbf{Output} & \textbf{Description} \\
			\hline
			0x20 & SHA3 & 30 & 2 & 1 & Compute a Keccak-256 hash. \\
			\hline
			\end{longtable}

        	\subsection{0x30's: Environmental Information}
			\begin{longtable}{|cP{2.8cm}cccp{6cm}|}
			\hline
	        	\textbf{Data} & \textbf{Opcode} & \textbf{Gas}  & \textbf{Input}  & \textbf{Output} & \textbf{Description} \\
			\hline
			0x30 & ADDRESS & 2 & 0 & 1 & Get the address of the currently executing account. \\
			0x31 & BALANCE & 400 & 1 & 1 & Get the balance of the given account. \\
			0x32 & ORIGIN & 2 & 0 & 1 & Get execution origination address. This is always the original sender of a transaction, never a contract account. \\
			0x33 & CALLER & 2 & 0 & 1 & Get caller address. This is the address of the account that is directly responsible for this execution. \\
			0x34 & CALLVALUE & 2 & 0 & 1 & Get deposited value by the instruction/transaction responsible for this execution. \\
			0x35 & CALLDATALOAD & 3 & 1 & 1 & Get input data of the current environment. \\
			0x36 & CALLDATASIZE & 2 & 0 & 1 & Get size of input data in current environment. This refers to the optional data field that can be passed with a message call instruction or transaction.\\
			0x37 & CALLDATACOPY & 3 & 3 & 0 & Copy input data in the current environment to memory. This refers to the optional data field passed with the message call instruction or transaction. \\
			0x38 & CODESIZE & 2 & 0 & 1 & Get size of code running in the current environment. \\
			0x39 & CODECOPY & 3 & 3 & 0 & Copy the code running in the current environment to memory. \\
			0x3a & GASPRICE & 2 & 0 & 1 & Get the price of gas in the current environment. This is the gas price specified by the originating transaction. \\
			0x3b & EXTCODESIZE & 700 & 1 & 1 & Get the size of an account's code. \\
			0x3c & EXTCODECOPY & 700 & 4 & 0 & Copy an account's code to memory. \\
			0x3d & RETURNDATASIZE & 2 & 0 & 1 & \\
			0x3e & RETURNDATACOPY & 3 & 3 & 0 & \\
			\hline
			\end{longtable}

	        \subsection{0x40's: Block Data}
			\begin{longtable}{|cP{2.8cm}cccp{6cm}|}
			\hline
			\textbf{Data} & \textbf{Opcode} & \textbf{Gas}  & \textbf{Input}  & \textbf{Output} & \textbf{Description} \\
			\hline
			0x40 & BLOCKHASH & 20 & 1 & 1 & Get the hash of one of the 256 most recent blocks. \footnote{A value of 0 is left on the stack if the block number is more than $256$ in number behind the current one, or if it is a number greater than the current one.} \\ 
			0x41 & COINBASE & 2 & 0 & 1 & Look up a block's beneficiary address by its hash.\\
			0x42 & TIMESTAMP & 2 & 0 & 1 & Look up a block's timestamp by its hash.\\
			0x43 & NUMBER & 2  & 0 & 1 & Look up a block's number by its hash. \\
			0x44 & DIFFICULTY & 2 & 0 & 1 & Look up a block's difficulty by its hash. \\
			0x45 & GASLIMIT & 2  & 0 & 1 & Look up a block's gas limit by its hash. \\
			\hline
			\end{longtable}

		\subsection{0x50's: Stack, memory, storage, and flow operations.}
			\begin{longtable}{|cP{2.8cm}cccp{6cm}|}
			\hline
			\textbf{Data} & \textbf{Opcode} & \textbf{Gas}  & \textbf{Input}  & \textbf{Output} & \textbf{Description} \\
			\hline
			0x50 & POP & 2 & 1 & 0 & Removes an item from the stack. \\
			0x51 & MLOAD & 3 & 1 & 1 & Load a word from memory. \\
			0x52 & MSTORE & 3 & 2 & 0 & Save a word to memory. \\
			0x53 & MSTORE8 & 3 & 2 & 0 & Save a byte to memory. \\
			0x54 & SLOAD & 200 & 1 & 1 & Load a word from storage. \\
			0x55 & SSTORE & 0 & 2 & 0 & Save a word to storage.\\
			0x56 & JUMP & 8 & 1 & 0 & Alter the program counter. \\
			0x57 & JUMPI & 10 & 2 & 0 & Conditionally alter the program counter. \\
			0x58 & PC & 2 & 0 & 1 & Look up the value of the program counter prior to the increment resulting from this instruction. \\
			0x59 & MSIZE & 2 & 0 & 1 & Get the size of active memory in bytes. \\
			0x5a & GAS & 2 & 0 & 1 & Get the amount of available gas, including the corresponding reduction for the cost of this instruction. \\
			0x5b & JUMPDEST & 1 & 0 & 0 & Mark a valid destination for jumps. \footnote{This operation has no effect on the \texttt{machine\_state during execution.}} \\
			\hline
			\end{longtable}

		\subsection{0x60-70's: Push Operations}
			\begin{longtable}{|cP{2.8cm}cccp{6cm}|}
        		\hline
        		\textbf{Data} & \textbf{Opcode} & \textbf{Gas}  & \textbf{Input}  & \textbf{Output} & \textbf{Description} \\
        		\hline
			0x60 & PUSH1 & - & 0 & 1 & Place a 1-byte item on the stack. \\
			0x61 & PUSH2 & - & 0 & 1 & Place a 2-byte item on the stack. \\
			0x62 & PUSH3 & - & 0 & 1 & Place a 3-byte item on the stack. \\
			0x63 & PUSH4 & - & 0 & 1 & Place a 4-byte item on the stack. \\
			0x64 & PUSH5 & - & 0 & 1 & Place a 5-byte item on the stack. \\
			0x65 & PUSH6 & - & 0 & 1 & Place a 6-byte item on the stack. \\
			0x66 & PUSH7 & - & 0 & 1 & Place a 7-byte item on the stack. \\
			0x67 & PUSH8 & - & 0 & 1 & Place a 8-byte item on the stack. \\
			0x68 & PUSH9 & - & 0 & 1 & Place a 9-byte item on the stack. \\
			0x69 & PUSH10 & - & 0 & 1 & Place a 10-byte item on the stack. \\
			0x6a & PUSH11 & - & 0 & 1 & Place a 11-byte item on the stack. \\
			0x6b & PUSH12 & - & 0 & 1 & Place a 12-byte item on the stack. \\
			0x6c & PUSH13 & - & 0 & 1 & Place a 13-byte item on the stack. \\
			0x6d & PUSH14 & - & 0 & 1 & Place a 14-byte item on the stack. \\
			0x6e & PUSH15 & - & 0 & 1 & Place a 15-byte item on the stack. \\
			0x6f & PUSH16 & - & 0 & 1 & Place a 16-byte item on the stack. \\
			0x70 & PUSH17 & - & 0 & 1 & Place a 17-byte item on the stack. \\
			0x71 & PUSH18 & - & 0 & 1 & Place a 18-byte item on the stack. \\
			0x72 & PUSH19 & - & 0 & 1 & Place a 19-byte item on the stack. \\
			0x73 & PUSH20 & - & 0 & 1 & Place a 20-byte item on the stack. \\
			0x74 & PUSH21 & - & 0 & 1 & Place a 21-byte item on the stack. \\
			0x75 & PUSH22 & - & 0 & 1 & Place a 22-byte item on the stack. \\
			0x76 & PUSH23 & - & 0 & 1 & Place a 23-byte item on the stack. \\
			0x77 & PUSH24 & - & 0 & 1 & Place a 24-byte item on the stack. \\
			0x78 & PUSH25 & - & 0 & 1 & Place a 25-byte item on the stack. \\
			0x79 & PUSH26 & - & 0 & 1 & Place a 26-byte item on the stack. \\
			0x7a & PUSH27 & - & 0 & 1 & Place a 27-byte item on the stack. \\
			0x7b & PUSH28 & - & 0 & 1 & Place a 28-byte item on the stack. \\
			0x7c & PUSH29 & - & 0 & 1 & Place a 29-byte item on the stack. \\
			0x7d & PUSH30 & - & 0 & 1 & Place a 30-byte item on the stack. \\
			0x7e & PUSH31 & - & 0 & 1 & Place a 31-byte item on the stack. \\
			0x7f & PUSH32 & - & 0 & 1 & Place a 32-byte item on the stack. \\			
			\hline
			\end{longtable}

		\subsection{0x80's: Duplication Operations}
			\begin{longtable}{|cP{2.8cm}cccp{6cm}|}
		        \hline
		        \textbf{Data} & \textbf{Opcode} & \textbf{Gas}  & \textbf{Input}  & \textbf{Output} & \textbf{Description} \\
		        \hline
			0x80 & DUP1 & - & 1 & 2 & Duplicate the 1st item in the stack. \\
			0x81 & DUP2 & - & 2 & 3 & Duplicate the 2nd item in the stack. \\
			0x82 & DUP3 & - & 3 & 4 & Duplicate the 3rd item in the stack. \\
			0x83 & DUP4 & - & 4 & 5 & Duplicate the 4th item in the stack. \\
			0x84 & DUP5 & - & 5 & 6 & Duplicate the 5th item in the stack. \\
			0x85 & DUP6 & - & 6 & 7 & Duplicate the 6th item in the stack. \\
			0x86 & DUP7 & - & 7 & 8 & Duplicate the 7th item in the stack. \\
			0x87 & DUP8 & - & 8 & 9 & Duplicate the 8th item in the stack. \\
			0x88 & DUP9 & - & 9 & 10 & Duplicate the 9th item in the stack. \\
			0x89 & DUP10 & - & 10 & 11 & Duplicate the 10th item in the stack. \\
			0x8a & DUP11 & - & 11 & 12 & Duplicate the 11th item in the stack. \\
			0x8b & DUP12 & - & 12 & 13 & Duplicate the 12th item in the stack. \\
			0x8c & DUP13 & - & 13 & 14 & Duplicate the 13th item in the stack. \\
			0x8d & DUP14 & - & 14 & 15 & Duplicate the 14th item in the stack. \\
			0x8e & DUP15 & - & 15 & 16 & Duplicate the 15th item in the stack. \\
			0x8f & DUP16 & - & 16 & 17 & Duplicate the 16th item in the stack. \\		
			\hline
			\end{longtable}
        
	
		\subsection{0x90's: Swap Operations}
			\begin{longtable}{|cP{2.8cm}cccp{6cm}|}
		        \hline  
		        \textbf{Data} & \textbf{Opcode} & \textbf{Gas}  & \textbf{Input}  & \textbf{Output} & \textbf{Description} \\
		        \hline  
			0x90 & SWAP1 & - & 2 & 2 & Exchange the 1st and 2nd stack items. \\
			0x91 & SWAP2 & - & 3 & 3 & Exchange the 1st and 3rd stack items. \\
			0x92 & SWAP3 & - & 4 & 4 & Exchange the 1st and 4th stack items. \\
			0x93 & SWAP4 & - & 5 & 5 & Exchange the 1st and 5th stack items. \\
			0x94 & SWAP5 & - & 6 & 6 & Exchange the 1st and 6th stack items. \\
			0x95 & SWAP6 & - & 7 & 7 & Exchange the 1st and 7th stack items. \\
			0x96 & SWAP7 & - & 8 & 8 & Exchange the 1st and 8th stack items. \\
			0x97 & SWAP8 & - & 9 & 9 & Exchange the 1st and 9th stack items. \\
			0x98 & SWAP9 & - & 10 & 10 & Exchange the 1st and 10th stack items. \\
			0x99 & SWAP10 & - & 11 & 11 & Exchange the 1st and 11th stack items. \\
			0x9a & SWAP11 & - & 12 & 12 & Exchange the 1st and 12th stack items. \\
			0x9b & SWAP12 & - & 13 & 13 & Exchange the 1st and 13th stack items. \\
			0x9c & SWAP13 & - & 14 & 14 & Exchange the 1st and 14th stack items. \\
			0x9d & SWAP14 & - & 15 & 15 & Exchange the 1st and 15th stack items. \\
			0x9e & SWAP15 & - & 16 & 16 & Exchange the 1st and 16th stack items. \\
			0x9f & SWAP16 & - & 17 & 17 & Exchange the 1st and 17th stack items. \\	
			\hline
			\end{longtable}

		\subsection{0xa0's: Logging Operations}
			\begin{longtable}{|cP{2.8cm}cccp{6cm}|}
		        \hline  
		        \textbf{Data} & \textbf{Opcode} & \textbf{Gas}  & \textbf{Input}  & \textbf{Output} & \textbf{Description} \\
		        \hline  
			0xa0 & LOG0 & 375 & 2 & 0 & Append log record with 0 topics. \\
			0xa1 & LOG1 & 750 & 3 & 0 & Append log record with 1 topic. \\
			0xa2 & LOG2 & 1125 & 4 & 0 & Append log record with 2 topic. \\
			0xa3 & LOG3 & 1500 & 5 & 0 & Append log record with 3 topic. \\
			0xa4 & LOG4 & 1875 & 6 & 0 & Append log record with 4 topic. \\
			\hline
			\end{longtable}

	        \subsection{0xf0's: System Operations}
			\begin{longtable}{|cP{2.8cm}cccp{6cm}|}
		        \hline  
	        \textbf{Data} & \textbf{Opcode} & \textbf{Gas}  & \textbf{Input}  & \textbf{Output} & \textbf{Description} \\
			\hline  
			0xf0 & CREATE & 32000 & 3 & 1 & Create a new contract account. Operand order is: value, input offset, input size. \\
			0xf1 & CALL & 700 & 7 & 1 & Message-call into an account. The operand order is: gas, to, value, in offset, in size, out offset, out size. \\
			0xf2 & CALLCODE & 700 & 7 & 1 & Message-call into this account with an alternative account's code. Exactly equivalent to CALL, except the recipient is the same account as at present, but the code is overwritten. \\
			0xf3 & RETURN & 0 & 2 & 0 & Halt execution, then return output data. This defines the output at the moment of the halt. \\
			0xf4 & DELEGATECALL & 700 & 6 & 1 & Message-call into this account with an alternative account's code, but with persisting values for \texttt{sender} and \texttt{value}. DELEGATECALL takes one less argument than CALL. This means that the recipient is in fact the same account as at present, but that the code is overwritten \textit{and} the context is almost entirely identical. \\
			0xf5 & CALLBLACKBOX & 40 & 7 & 1 & - \\
			0xfa & STATICCALL & 40 & 6 & 1 & - \\
			0xfd & REVERT & 0 & 2 & 0 & - \\
			0xfe & INVALID & - & 1 & 0 & Designated invalid instruction. \\
			0xff & SELFDESTRUCT & 5000 & 1 & 0 & Halt execution and register the account for later deletion. \\
			\hline
			\end{longtable}

		
	\section{Higher Level Languages}
		\subsection{Lower-Level Lisp}
			The Lisp-Like low level language: a human-writable language used for authoring simple contracts and trans-compiling to higher-level languages.
	
		\subsection{Solidity}
			A language similar in syntax to Javascript, and the most commonly used language for creating smart contracts in Ethereum.
		\subsection{Serpent}
			A deprecated language.
		\subsection{Vyper}
			A newer language for developing smart contracts -- still under development.



\clearpage
\begin{multicols*}{2}
\printbibliography
\clearpage
\printglossary[type=\acronymtype]
\glsaddall
\printnoidxglossaries
\clearpage
\end{multicols*}
\end{alphafootnotes}

\printindex

\end{document}
